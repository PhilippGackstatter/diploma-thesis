\chapter{Conclusion}
\label{chapter:conclusion}

"In principle, its scale-to-zeroproperty (i.e., unused containers are deallocated from the platform)suits very well energy-aware IoT use cases with intermittent ap-plications" \cite{Aslanpour2021}

-> Less cold start latency means less time keeping containers alive (mentioned in background - serverless workload).

Other attempts for speeding-up serverless have been successful, but didn't catch on. We assume this is due to custom-written application frameworks that aren't industry-proven. This is different in our case, since the public-facing interface that serverless developers need to interact with is WebAssembly, whose adoption rate is increasing.

\citeauthor{Shahrad2020} have found that, on average, 50\% of functions execute for more than 1s, that 80\% of functions execute for more than 100ms, and 9X\% of functions execute for more than 10ms (or similar, perhaps need to do the actual calculations on the data set). Thus our cold start latencies are not on the same order of magnitude as 9X\% of function execution times, but faster.

\section{Open challenges}

- multi threading not in Wasm
- limiting CPU time (research if possible)
- limiting memory
