\chapter{Introduction}
\label{chapter:introduction}

\section{Motivation}

With the growing number of Internet of Things (IoT) devices, the limitations of today's cloud computing model have become apparent. Low latency and data-intensive use cases cannot be facilitated by the cloud, due to its physical distance and the sheer amount of data that all edge devices would have to transmit over a very finite amount of bandwidth \cite{Aslanpour2021}.
Hence, application scenarios at the edge involving data analytics \cite{Nastic2017}, machine learning \cite{Rausch2019} or augmented reality \cite{Baresi2019}, require moving away from the cloud-centric model.
The edge computing paradigm has emerged as a solution to address these issues.
Applying the traditional cloud model to the edge, however, does not promise to be an effective solution. Due to the limited resources available, allocating resources like virtual machines or even containers in a fixed manner may prove to be too inefficient \cite{Baresi2019}.
Instead, a model with a higher resource elasticity is required; one that can scale up and down fast, thereby efficiently using the available resources.
One approach that implements this is Serverless computing -- most commonly in the form of Function-as-a-Service (FaaS).
In this computing model, developers can execute functions without having to specify how the necessary infrastructure is set up. Instead, the allocation of computing resources is automated, which enables a higher elasticity. This property is well-suited to the resource- and energy-constrained environments of edge devices, given that overprovisioning is even less desirable than in the cloud model \cite{Aslanpour2021}.
Because resources can be utilized whenever required, this model is ideal for event-driven architectures, such as those typically found in IoT scenarios, where the time at which devices require computing resources is unpredictable, e.g. when a sensor receives input or a user presses a button. Therefore, it may happen that at times, a serverless framework receives only a few requests, while at others, it needs to process a large amount of requests, simultaneously.
To enable this elasticity, one user's function might run on the same host as that of another. To isolate these instances, serverless frameworks make use of OS-level virtualization through containers -- usually of the Docker flavor.
Next to security, it also enables packaging language runtimes, so developers can write functions almost language-agnostic, and statelessness due to cheap create and destroy operations. However, cheap is relative. Historically, containers have been compared to the heavier-weight virtual machines, where the creation is indeed much more time intensive. However, on the time scale of serverless, where an incoming request ought to be finished as soon as possible, but the container needs to be created first, it turns out to be not so cheap after all. Not pre-allocating resources to avoid overprovisioning is the point of serverless, but that presupposes that the allocation itself is timely.

When a function is first executed, its container must be started -- referred to as the cold start. This can introduce a latency of a few hundred milliseconds to even seconds \cite{Manner2018, Wang2018}. Consider that on average, 50\% of serverless functions execute for less than one second \cite{Shahrad2020}. Thus, the latency observed by the end user can often be double of what the function itself is responsible for. This infrastructure-induced latency is unacceptable for time-sensitive applications, and thereby most user-facing ones.
To make matters worse, these cold start latencies increase under concurrent workloads -- precisely those supposed to be handled well by serverless platforms \cite{Mohan2019}. Containers were not designed for this use case and the startup time they require is inherent to how they achieve isolation.
This problem has been partially addressed by keeping containers warm in-between requests; that is, only the first of potentially many invocations incur this cost. However, this is a form of overprovisioning -- though a less severe one -- and therefore opposed to the serverless promise of being able to scale to zero.
Evidently, this is already a significant problem for the cloud, so applying this approach at the resource-constrained edge, like Amazon Web Services' (AWS) Greengrass\footnote{\url{https://aws.amazon.com/greengrass/}} would only exacerbate the issue.
As \citeauthor{Nastic2018} point out, the »architectural and design assumptions behind such approaches need to be reevaluated« \cite{Nastic2018}. Consequently, to rectify this problem, a new approach is needed.

% Thus, a serverless computing platform only needs to provide a Wasm runtime for the instruction set architectures of its edge devices.

At the heart of the problem lies the container runtime. A new approach needs to provide similar functionality and guarantees, while offering a more efficient cold start. One nascent technology that could facilitate this radical replacement is called WebAssembly (Wasm).
WebAssembly is a portable, binary instruction format for memory-safe, sandboxed execution in a virtual machine \cite{W3C2020}. Its portability means that the compiled WebAssembly modules can be executed wherever a runtime exists.
Thus, a function can easily be moved between architecturally heterogeneous devices. Its sandboxing features enable Wasm to be used for the execution of arbitrary, user-supplied code without compromising the security of a multi-tenant system. Multiple compilation strategies exist for running WebAssembly, such that it can reach near-native speeds. This is important since once Docker containers are running, they can execute at native speeds or as fast as the language runtime can offer. In the process of alleviating the cold start, we want to avoid suffering a massive loss in execution performance. Finally, these modules can be created and destroyed in a matter of microseconds. The cloud provider fastly has begun offering execution of Wasm code on its FaaS platform through the use of their \inl{lucet} runtime. In its announcement, they write:

\begin{quote}
  »Lucet can instantiate WebAssembly modules in under 50 microseconds, with just a few kilobytes of memory overhead. By comparison, Chromium’s V8 engine takes about 5 milliseconds, and tens of megabytes of memory overhead, to instantiate JavaScript or WebAssembly programs« \cite{fastly2019}.
\end{quote}

This gives us a first indication that WebAssembly runtimes does not suffer from a cold start on the same order of magnitude as Docker's.
In summary, Wasm allows for secure, portable and fast execution. With these properties it may be a good replacement for containers on edge computers. One of the founders of Docker even commented:

\begin{quote}
  »If WASM+WASI existed in 2008, we wouldn't have needed to created Docker. That's how important it is. Webassembly on the server is the future of computing« \cite{Hykes2019}.
\end{quote}

Hence, WebAssembly's potential for serverless computing shall be examined more closely in this work.

% To enable low-latency edge computing on resource-constrained devices, serverless is an ideal framework. Single functions are lightweight enough to be deployed on these devices, where entire applications are not. But executing functions in container runtimes on resource-constrained devices will only exacerbate the cold-start problem. Thus, a more scalable and lightweight runtime is needed. One way to achieve this, is to replace the container runtime with one based on WebAssembly (Wasm) such as in \cite{Hall2019} and \cite{Murphy2020}.

% \begin{itemize}
%   \item IoT devices which consume edge resources frequently, are often event-driven. This is a great fit for serverless, as the function can be called exactly when it is needed, provisioned by the platform, then destroyed
%   \item While Docker introduces almost no overhead for cpu-intensive workloads, the cold-start problem may be exacerbated by the less powerful hardware
% \item In containers, the runtime for an interpreted language such as Python or JavaScript needs to be shipped. If we can compile these languages to Wasm, then no additional runtime needs to be shipped. This is code that doesn't need to be downloaded on every invocation from the database.
% \end{itemize}

% Since the edge computing paradigm is already well-aligned with cloud-based serverless computing, but lacks a solution that works well on limited resources, a solution that works well on edge devices should also be suitable for use in cloud-based environments. Serverless functions should already be as lightweight as possible in both scenarios, so the goals for both are well-aligned.
% The current solution is not applicable, only because it is too heavyweight, which is \emph{also} a problem for the cloud itself. Thus sth that works well for the edge ... Instead of applying the cloud model to the edge, apply the edge to the cloud.


\section{Aim of the Work}

The goal of this work is to evaluate the viability of WebAssembly in serverless computing in the context of edge as well as cloud computing. We assess that potential by gathering requirements from the current approaches to serverless, as well as those from edge computing, which supersede the current ones. Based on those, we study several WebAssembly runtimes for their suitability towards our goals. We then discuss two approaches, which would enable WebAssembly to be used in serverless, and implement one of them. Given that WebAssembly runtimes themselves are geared towards different use cases, we leverage three of them and aim to find the one most suitable for our needs. Ultimately, the work is guided by and seeks to answer the following research questions.

\subsubsection*{By how much can WebAssembly runtimes alleviate the cold start latency?}

The first research question examines whether our core hypothesis holds. That is, can our serverless container runtime provision WebAssembly containers more quickly than the traditional Docker runtime can? If so, how much faster, and with what optimizations in place? We evaluate that by measuring the cold start latency for both approaches on an edge as well as a server-grade device.

\subsubsection*{How do the performance characteristics of the WebAssembly and Docker container runtimes differ?}

The goal of the second question is to measure the performance of both container technologies at runtime. More specifically, at what speed can certain workload types be executed? We run CPU- and I/O-bound requests in isolation, to test performance in these specific aspects, but also mixed workloads to test scenarios closer to reality. Moreover, we take the cold start into account in some test runs while isolating it in others to give a holistic picture.

\subsubsection*{What are the resource costs for keeping containers warm in both container runtimes?}

The third question explores, what the cost of not scaling to zero is, i.e. when containers are kept warm. This is reflected in the memory usage of such containers, which are being kept ready for potential incoming requests. That ties in with the cost of cold starts. If cold starts are cheap, we can reduce or eliminate the time that containers need to be kept warm. Orthogonally, if keeping containers warm is cheaper with WebAssembly than Docker, keeping more of them around is feasible to improve performance overall. Both measurements let us explore the space of configurations this keep-alive optimization has and compare it to that of Docker.

\subsubsection*{Which WebAssembly runtime is the most suitable for use on edge devices?}

The fourth question is directed at finding the WebAssembly runtime best suited for the execution of modules on low-energy and resource-constrained edge devices. We use a Raspberry Pi single-board computer as our edge device to run our serverless environment on and evaluate the different runtimes from there. All of the preceding measurements as well as our experiences in the design phase are amalgamated to produce an answer to this question.

\subsection{Terminology}

First, a note on terminology for the rest of this work. Since runtime is a very overloaded term in our context, we may also refer to container runtimes as executors, i.e. the programs responsible for managing containers. Runtime appears also as the word for language runtimes, for example, \inl{node.js} for JavaScript or a Python interpreter.
WebAssembly runtimes are virtual machines executing WebAssembly code, which is called a WebAssembly module. A WebAssembly module in the context of those runtimes' API means the in-memory representation of such a module. Finally, a WebAssembly container is a more generic term encompassing both a WebAssembly module and the instance produced from it, which is the module in execution.

\subsection{Contributions}

The first contribution is a review of the literature, to provide a background for serverless computing in the cloud and at the edge as well as a comprehensive overview of WebAssembly and its salient properties, which makes it a viable alternative to Docker for our use case. That includes an assessment of the current research space as to what constitutes serverless use cases and typical workloads, in order to facilitate a realistic evaluation of our system.

The major contribution revolves around modifying an open-source serverless framework, Apache OpenWhisk, to enable the use of such a new container runtime. We explore two auspicious options for bringing WebAssembly support to OpenWhisk and settle for one. At the heart of the work is the implementation of the container runtime using three different WebAssembly runtimes. We discuss the properties the executor needs to fulfill and the design that aims to implement them. Subsequently, we evaluate the three resulting executors against the baseline OpenWhisk vanilla for the cold start latency, the performance under different workload types as well as memory usage for warm containers. Finally, we explore alternative approaches to alleviate the cold start problem and conclude with answers to the research questions we posed.

\section{Structure of the Work}

In Chapter \ref{chapter:background} we expand on the introduced topics, to establish a common background for the remainder of the work, particularly the serverless workload and WebAssembly.

In Chapter \ref{chapter:design} we discuss approaches for potential solutions to our problem and discuss the implementation of our proposed solution. We settle on one approach and implement it, work out the differences between the WebAssembly runtimes we leverage, and discuss trade-offs along the way.

In Chapter \ref{chapter:evaluation} we evaluate the system for the cold start latency, different workload types and memory usage in warm containers.

In Chapter \ref{chapter:relatedwork} we examine the current research for other approaches that attempt to solve the cold start problem and discuss the differences to our approach. \todo{And possibly more... will update once chapter is written.}

In Chapter \ref{chapter:conclusion} we summarize the results of our work and give an overview of open research questions.