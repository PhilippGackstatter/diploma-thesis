\chapter{Introduction}
\label{chapter:introduction}

\definecolor{BrightOrange}{HTML}{8c14ee}
\definecolor{DarkOrange}{HTML}{401159}

\usemintedstyle{friendly}

% Minted Inline
\definecolor{friendlybg}{HTML}{f0f0f0}
\newcommand{\inl}[1]{\mintinline[style=friendly,bgcolor=friendlybg]{shell}|#1|}

\newcommand{\quot}[1]{\foreignquote{german}{#1}}

\section{Motivation}

With the growing number of Internet of Things (IoT) devices, the limitations of today's cloud computing platforms have become apparent. Supporting low latency in use cases like data analytics and machine learning requires moving away from the cloud-centric model \cite{Nastic2017}, \cite{Rausch2019}. Serverless computing, or Function-as-a-Service (FaaS), has emerged as a solution.

\begin{quote}
  \quot{Serverless computing is an emerging cloud-based execution model in which user-defined functions are seamlessly and transparently hosted and managed by a distributed platform.} \cite{Nastic2017}
\end{quote}

In this computing model, cloud service providers offer to execute functions written in a programming language of the developer's choice. This is most often facilitated by the use of a container technology, such as Docker. The container enables practical execution of any program, as it contains all its dependencies. Through this mechanism, it also isolates one executing function from another to securely enable multi-tenancy on shared physical hosts. While this works well on the level of Software-as-a-Service, it has problems on the finer-grained Function-as-a-Service (FaaS) level. Once containers are running, they allow execution of programs at native speeds. However, in serverless environments, the need to quickly create and destroy functions arises. When a function is first executed, its container must be started. This can introduce latency in the order of a few hundred milliseconds to seconds \cite{Manner2018} and gets exponentially worse under concurrent workloads \cite{Mohan2019}. The setup of the execution environment may thus contribute a significant part to the end-to-end time it takes to invoke a function for the first time. This has been coined the cold-start problem.
% This problem has been partially addressed by keeping containers \quot{warm} in-between requests. However this is opposed to the serverless promise of being able to scale to zero.


\medskip

To enable low-latency edge computing on resource-constrained devices, serverless is an ideal framework. Single functions are lightweight enough to be deployed on these devices, where entire applications are not. But executing functions in container runtimes on resource-constrained devices will only exacerbate the cold-start problem. Thus, a more scalable and lightweight runtime is needed. One way to achieve this, is to replace the container runtime with one based on WebAssembly (Wasm) such as in \cite{Hall2019} and \cite{Murphy2020}. Wasm is a portable, binary instruction format for memory-safe, sandboxed execution in a virtual machine \cite{W3C2020}. Its portability means that it can be executed wherever an executor exists. Thus, a serverless computing platform only needs to provide a Wasm runtime for the instruction set architectures of its edge devices. Then, a function can be easily moved between architecturally heterogeneous devices. Its sandboxing features enables Wasm to be used for the execution of arbitrary, user-supplied code without compromising the security of a multi-tenant system. Finally, Wasm executors can be created and destroyed in a matter of microseconds. The cloud provider Fastly has recently begun offering execution of Wasm modules on its FaaS platform through the use of their \inl{lucet} executor. On their blog they write:

\begin{quote}
  \quot{Lucet can instantiate WebAssembly modules in under 50 microseconds, with just a few kilobytes of memory overhead. By comparison, Chromium’s V8 engine takes about 5 milliseconds, and tens of megabytes of memory overhead, to instantiate JavaScript or WebAssembly programs.} \cite{fastly2019}
\end{quote}

Thus, Wasm runtimes may not suffer from a noticeable cold-start.
In summary, Wasm allows for secure, portable and fast execution. With these properties it may be a good replacement for containers on edge computers. One of the founders of Docker even commented:

\begin{quote}
  \quot{If WASM+WASI existed in 2008, we wouldn't have needed to created Docker. That's how important it is. Webassembly on the server is the future of computing.} \cite{Hykes2019}
\end{quote}

% \begin{itemize}
%   \item IoT devices which consume edge resources frequently, are often event-driven. This is a great fit for serverless, as the function can be called exactly when it is needed, provisioned by the platform, then destroyed
%   \item While Docker introduces almost no overhead for cpu-intensive workloads, the cold-start problem may be exacerbated by the less powerful hardware
% \item In containers, the runtime for an interpreted language such as Python or JavaScript needs to be shipped. If we can compile these languages to Wasm, then no additional runtime needs to be shipped. This is code that doesn't need to be downloaded on every invocation from the database.
% \end{itemize}

This promising outlook shall be examined more closely in this work. We will implement a Wasm runtime, using different Wasm executors, in Apache OpenWhisk and compare it to the existing container runtime.

\section{Aim of the Work}

The goal of this work is to evaluate the viability of WebAssembly in serverless computing in the context of edge computing. We asses that by implementing a Wasm runtime for serverless computing to answer these research questions.

\begin{enumerate}
  % \item What are the challenges in adding a Wasm runtime to OpenWhisk? 

  % \item What are the benefits and drawbacks of a WebAssembly runtime in OpenWhisk?

  % \item What are the differences in creating a function using WebAssembly rather than containers?

  % \item How suitable are the various WebAssembly runtimes for the use on heterogeneous edge computers?
  % Which of the evaluated WebAssembly runtimes is the most appropriate one for the use on edge computers / in a heterogeneous environment?
  % \begin{itemize}
  %   \item Performance
  %   \item heterogeneous -> Architecture support
  %   \item heterogeneous -> Which one scales from a raspi pi to a powerful cloudlet?
  % \end{itemize}

  \item How suitable are the runtimes for the use on heterogeneous edge computers?

  \item How do the performance characteristics of the Wasm and container runtimes differ?

  \item By how much can the cold-start problem be alleviated under the Wasm runtimes?

        %   \item How do the wasm-based runtimes fare against the existing container solution in OpenWhisk? Can the cold-start problem be alleviated? Are there use-cases where containers are still more appropriate?
        %   \begin{itemize}
        %     \item Performance comparison
        %     \item cold-start?
        %     \item for highly-invoked functions, maybe containers are still better due to native speed. But lucet may be a serious contender here.
        %     \item If something like JavaScript can actually be compiled to wasm and then to native code, it may be a lot faster than running in a node.js environment. So there may even be potential for interpeted languages to gain performance!
        %   \end{itemize}
\end{enumerate}